%%%%%%%%%%%%%%%%%%%%%%%%%%%%%%%%%%%%%%%%%
% Twenty Seconds Resume/CV
% LaTeX Template
% Version 1.1 (8/1/17)
%
% This template has been downloaded from:
% http://www.LaTeXTemplates.com
%
% Original author:
% Carmine Spagnuolo (cspagnuolo@unisa.it) with major modifications by 
% Vel (vel@LaTeXTemplates.com)
%
% License:
% The MIT License (see included LICENSE file)
%
%%%%%%%%%%%%%%%%%%%%%%%%%%%%%%%%%%%%%%%%%

%----------------------------------------------------------------------------------------
%	PACKAGES AND OTHER DOCUMENT CONFIGURATIONS
%----------------------------------------------------------------------------------------

\documentclass[letterpaper]{twentysecondcv} % a4paper for A4

%----------------------------------------------------------------------------------------
%	 PERSONAL INFORMATION
%----------------------------------------------------------------------------------------

% If you don't need one or more of the below, just remove the content leaving the command, e.g. \cvnumberphone{}

\profilepic{image.jpg} % Profile picture

\cvname{} % Your name
\cvjobtitle{Mochammad Fatchur Rahman} % Job title/career

\cvdate{Rembang, 07 July 1994} % Date of birth 
\cvaddress{Blok A, South Jakarta} % Short address/location, use \newline if more than 1 line is required
\cvnumberphone{+6289527942835} % Phone number
\cvsite{} % Personal website
\cvmail{fatchur.rahman1@gmail.com } % Email address

%----------------------------------------------------------------------------------------

\begin{document}

%----------------------------------------------------------------------------------------
%	 ABOUT ME
%----------------------------------------------------------------------------------------

\aboutme{} % To have no About Me section, just remove all the text and leave \aboutme{}

%----------------------------------------------------------------------------------------
%	 SKILLS
%----------------------------------------------------------------------------------------

% Skill bar section, each skill must have a value between 0 an 6 (float)
\skills{{Computer Vision/4},{Go/3}, {Python Backend (Flask)/5.5}, {Data Science/5}, {ML Engineering/5}, {AWS (Sagemaker, lambda, s3, SNS, etc.)/5}, {GCP (GAE, GKE, AI-platform, etc.)/5}, {Precision Farming/4}, {Teaching/4}}

%------------------------------------------------+++


\makeprofile % Print the sidebar
%----------------------------------------------------------------------------------------
%	 EDUCATION
%----------------------------------------------------------------------------------------

\section{Educations}

\begin{twenty} % Environment for a list with descriptions
	\twentyitem{2012 - 2016}{IPB University {\normalfont }}{}{Department of Mechanical and Biosystem Engineering \\ 
	Faculty of Agricultural Engineering and Technology \\
	Thesis: Vision-Based Pineapple Detection and Ripeness Classification \\
	Thesis Link:  \href{https://repository.ipb.ac.id/handle/123456789/85989}{IPB repository} \\
    Thesis Video: \href{https://www.youtube.com/watch?v=d7238va97eQ&t=50}{video link} }
	%\twentyitem{<dates>}{<title>}{<location>}{<description>}
\end{twenty}

%----------------------------------------------------------------------------------------
%	 AWARDS
%----------------------------------------------------------------------------------------

\section{Awards}

\begin{twentyshort} % Environment for a short list with no descriptions
	\twentyitemshort{2017}{The 3rd winner of go-hackaton 2017, held by Go-jek (Indonesia first unicorn startup)}
	%\twentyitemshort{2013 - 2016}{Department Award}{\normalfont }
	\twentyitemshort{2013}{GPA 4.0/4.0 TPB IPB (1st - 2nd semesters)}
\end{twentyshort}



%----------------------------------------------------------------------------------------
%	 PUBLICATIONS
%----------------------------------------------------------------------------------------

\section{Experiences}

\begin{twentyshort} % Environment for a short list with no descriptions
	\twentyitemshort{Feb 2019 - present}{ AI Engineer, Qoala.id}
	\twentyitemshort{Sept 2017 - Oct 2018}{ AI Engineer, Nodeflux.io}
	\twentyitemshort{Nov 2016 - Sept 2017}{ RnD staff, ar-innovation.com}{\normalfont }
	%\twentyitemshort{<dates>}{<title/description>}
\end{twentyshort}


%----------------------------------------------------------------------------------------
%	 Repository
%----------------------------------------------------------------------------------------
\section{Git and Others}

\begin{twenty}
	\twentyitemshort{Linkedin}{\href{https://www.linkedin.com/in/mochammad-fatchur-rahman-a48137a8/} {linkedin profile}{\normalfont }}{}{}
	\twentyitemshort{Github}{\href{https://github.com/fatchur} {github profile link}{\normalfont }}{}{}
	\twentyitemshort{Pypi}{\href{https://pypi.org/user/fatchur/} {python repository profile}{\normalfont }}{}{}
\end{twenty}


%----------------------------------------------------------------------------------------
%	 VOLUNTEERS
%----------------------------------------------------------------------------------------

\section{Volunteers}

\begin{twenty} % Environment for a short list with no descriptions
	\twentyitem{03/2020}{
		Activity: Juara GCP AI Workshop, Tutor \\
		Description: Workshop on google cloud AI deployment (AI platform and dataflow/apache beam)}{}{Machine Learning Indonesia Community}
	\twentyitem{11/2019}{
		Activity: Tensorflow World Extended Jakarta, Speaker\\
		Description:  Talking about tensorflow deployment (ML-Engine, tensorflow lite, and tensorflow serving)}{}{Machine Learning Indonesia Community}
	\twentyitem{11/2019}{
		Activity: Machine Learning on Google Cloud Workshop, Tutor.\\
		Description:  Workshop on machine learning models deployment in GCP (especially for agriculture projects like weeds detection, fruit grading, etc)}{}{Dept of Mechanical and Biosystem Engineering, IPB}
	\twentyitem{09/2019}{
		Activity: Telkom Indonesia Tensorflow Workshop, Tutor.\\
		Description:  Workshop on tensorFlow serving, google AI platform, pub/sub, app-engine, and the cloud function}{}{Telkom Indonesia Data Science (Vision) Division}
	\twentyitem{08/2018}{
		Activity: Data Science Indonesia MFM Program, Tutor.\\
		Description: Workshop on digital image processing}{}{Data Science Indonesia Community}
	\twentyitem{04/2018}{
		Activity: Data Science Weekend, Tutor.\\
		Description: Workshop on deep learning with TensorFlow}{}{Data Science Indonesia Community x Multimedia Nusantara University}
	\twentyitem{2013 - 2014}{
		Activity: Bidik Misi Scholarship Tutorial, Tutor.\\
		Description: Free tutorial for bidikmisi awardee}{}{Bidik Misi Scholarship Community\\
		IPB University}{\normalfont}
	%\twentyitemshort{<dates>}{<title/description>}
\end{twenty}

%----------------------------------------------------------------------------------------
%	 EXPERIENCE
%----------------------------------------------------------------------------------------

\section{Projects}

\begin{twenty}
	\twentyitem{Qoala.id}{[2019-] Computer Vision for Automated Claim}{}{We are working with some computer vision algorithms like object detection, segmentation, classification, keypoint detection, and face recognition for building a system of automate insurance assesment by videos or images. Our products are: \\
		
	1. Automate assesment of smartphone screen protection resgistration using video \\
    2. Automate assesment of vehicle body protection resgistration using images \\
	3. South East Asia ID Cards OCR and face comparison \\

    All of  our researches are already integrated into Qoala ecosystem and realesed as products. For the technologies, we are using both GCP (app-engine, AI-platform, kubernetes engine, pub/sub, storage, datastore, bigquery, mysql, GCR, and dataflow) and AWS (Lambda, gateway, sagemaker, s3, ECR, and SNS). Currently, we also finish the development of our automate model training for both GCP and AWS. }\\
	
	
	\twentyitem{-Hobby-}{[2018-2019] Building a simple-tensor: Simplification of tensorflow operations}{}{
	This project is to create an opensource python package for simplifying the usage of tensorflow, especially for tensor operations, losses, and architectures. Our package also provides some computer vision algorithms, ex:\\
    
    1. Object detection(YOLO-V3) module (built from scratch): "simple-tensor" allows you to modify the size of your YOLO-V3 model up to 30 times smaller or 6.7 MB than the original (>200MB).

    2. Image classification module: A simplification of the tensorflow image classification. It depends on tensorflow slim as base architectures. Currently, "simple-tensor" only support for resnet, densenet, and inception-v4.
    
    3. Segmentation module: Deepl-lab \\
    
    Github: \href{https://github.com/fatchur/Simple-Tensor} {Github Link}\\
    Pypi: \href{https://pypi.org/project/simple-tensor/} {How to install}\\
    }\\

	
	\twentyitem{Side Job}{[2019] PT. Pertamina, Unit Balongan Refinery, Vision Based Worker Monitoring.}{}{We developed a vision-based video analysis for Balongan Refinery workers. The system detects violations, ex: not using helm or standard jacket.  
	
	This project depends on some technologies:
	
	1. Opencv: Streaming the video from CCTV, manipulating the image.
	
	2. Kafka: Streaming the data between containers
	
	3. Redis: Saving the temporary data, ex: camera URL
	
	4. Tensorflow: Inferencing the input image
	
	5. Gunicorn, Gevent, Flask: Webserver
	
	6. Flask-SocketIO: Broadcasting the resulting image to the users
	
	7. Mysql: DB for the detected violations
	
	8. Docker}

\end{twenty}

\begin{twenty} % Environment for a list with descriptions
   	\twentyitem{Nodeflux.io}{[2018] Small Image Reconstruction}{}{Reconstruction of a low-resolution image to be a high-resolution with a generative adversarial network (part of deep learning), far better than the OpenCV resize}
	
	\twentyitem{Nodeflux.io}{[2018] License Plate Recognition}{}{A project for ASIAN Games and IMF Bali Meeting 2018 in Jakarta, Palembang, and Bandung.}
	
	\twentyitem{Go-Hackathon}{[2016] Indonesian Food Images Recognition with Deep Learning}{}{Project video: \href{https://www.youtube.com/watch?v=G2WHTZhY-tU}{Youtube video link}}
	
	\twentyitem{WIR Group}{[2017] Face Recognition}{}{}
	
	\twentyitem{WIR Group}{[2016] Face Gender Recognition}{}{
		 We worked on project for recognizing the gender of the user}
	
	\twentyitem{WIR Group}{[2016] Face Expression Recognition}{}{We did a research to recognize the face expression of the user}
	
	\twentyitem{IPB}{[2015] Vision Based Pineapple Detection and Quality Evaluation }{}{Research video: \href{https://www.youtube.com/watch?v=d7238va97eQ&t=50}{Youtube video link}}
	
	%\twentyitem{<dates>}{<title>}{<location>}{<description>}
\end{twenty}

%----------------------------------------------------------------------------------------
%	 OTHER INFORMATION
%----------------------------------------------------------------------------------------


%----------------------------------------------------------------------------------------
%	 SECOND PAGE EXAMPLE
%----------------------------------------------------------------------------------------

%\newpage % Start a new page

%\makeprofile % Print the sidebar

%\section{Other information}

%\subsection{Review}

%Alice approaches Wonderland as an anthropologist, but maintains a strong sense of noblesse oblige that comes with her class status. She has confidence in her social position, education, and the Victorian virtue of good manners. Alice has a feeling of entitlement, particularly when comparing herself to Mabel, whom she declares has a ``poky little house," and no toys. Additionally, she flaunts her limited information base with anyone who will listen and becomes increasingly obsessed with the importance of good manners as she deals with the rude creatures of Wonderland. Alice maintains a superior attitude and behaves with solicitous indulgence toward those she believes are less privileged.

%\section{Other information}

%\subsection{Review}

%Alice approaches Wonderland as an anthropologist, but maintains a strong sense of noblesse oblige that comes with her class status. She has confidence in her social position, education, and the Victorian virtue of good manners. Alice has a feeling of entitlement, particularly when comparing herself to Mabel, whom she declares has a ``poky little house," and no toys. Additionally, she flaunts her limited information base with anyone who will listen and becomes increasingly obsessed with the importance of good manners as she deals with the rude creatures of Wonderland. Alice maintains a superior attitude and behaves with solicitous indulgence toward those she believes are less privileged.

%----------------------------------------------------------------------------------------

\end{document} 
